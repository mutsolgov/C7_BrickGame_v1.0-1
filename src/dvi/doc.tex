\documentclass[a4paper,12pt]{article}

\usepackage[russian]{babel}
\usepackage{graphicx}

\title{Документация к игре Tetris}
\author{Разработчик}
\date{\today}

\begin{document}

\maketitle

\section{Введение}
Игра Tetris представляет собой классическую головоломку, в которой игроку необходимо складывать падающие фигуры (тетромино) так, чтобы они заполнили горизонтальные линии.

\section{Правила игры}
Фигуры падают сверху вниз, и игрок может:
\begin{itemize}
    \item Перемещать фигуру влево и вправо
    \item Поворачивать фигуру
    \item Ускорять падение фигуры
    \item Приостанавливать игру
\end{itemize}

Полностью заполненная горизонтальная линия исчезает, и игрок получает очки.

\section{Управление}
\begin{itemize}
    \item \textbf{Стрелка влево} - сдвиг фигуры влево
    \item \textbf{Стрелка вправо} - сдвиг фигуры вправо
    \item \textbf{Стрелка вниз} - ускорение падения
    \item \textbf{Пробел} - поворот фигуры
    \item \textbf{P} - пауза
    \item \textbf{Esc} - выход
\end{itemize}

\section{Алгоритмы игры}
\subsection{Обновление состояния игры}
Каждый кадр игра проверяет текущее состояние:
\begin{itemize}
    \item Если фигура достигает нижней границы или касается другой фигуры, она фиксируется на поле.
    \item Проверяются заполненные линии и, если таковые имеются, они удаляются.
    \item Создается новая фигура и помещается в верхней части экрана.
\end{itemize}

\subsection{Обработка ввода пользователя}
Программа обрабатывает ввод с клавиатуры и изменяет положение или ориентацию фигуры.

\section{Компиляция и запуск}
Для сборки используйте команду:
\begin{verbatim}
make all
\end{verbatim}

Для генерации документации в формате DVI:
\begin{verbatim}
make dvi
\end{verbatim}

\end{document}
